%%%% Paramétrage du cours %%%%
\def\xxactivite{Cours}
\def\xxauteur{\textsl{Xavier Pessoles}}

\fichefalse
\proftrue
\tdfalse
\courstrue

\def\xxnumchapitre{Chapitre 1 \vspace{.2cm}}
\def\xxchapitre{\hspace{.12cm} Introduction}

\def\xxcompetences{%
\textsl{%
\textbf{Savoirs et compétences :}\\
\begin{itemize}[label=\ding{112},font=\color{ocre}] 
\item A VOir
%\item \textit{Mod2.C17} : torseur dynamique
%\item \textit{Mod2.C17.SF1} : déterminer le torseur dynamique d’un solide, ou d’un ensemble de solides, par rapport à un autre solide
%\item \textit{Mod2.C15} : matrice d'inertie
%\item \textit{Res1.C2} : principe fondamental de la dynamique
%\item \textit{Res1.C1.SF1} : proposer une démarche permettant la détermination de la loi de mouvement
%\item \textit{Res1.C2.SF1} : proposer une méthode permettant la détermination d’une inconnue de liaison
\end{itemize}
}}



\def\xxfigures{
}%figues de la page de garde

\iflivret
\input{../../style/new_pagegarde}
\else
\input{../../style/new_pagegarde}
\fi
\setlength{\columnseprule}{.1pt}

\vspace{2cm}
\pagestyle{fancy}
\thispagestyle{plain}


\section{Définitions}

\begin{itemize}
\item Data scientist
\item machine learning
\item deep learning
\item réseaux de neurons
\item régression
\item data mining
\item bigdata
\item données continues, données discrètes, données nominales, données ordinales, données (semi-)structurées et non structurées

\end{itemize}

\begin{itemize}
\item algorithmes supervisés, non supervisés
\item algorithmes de régression et de classification
\end{itemize}



\begin{thebibliography}{2}
   \bibitem[1]{ref1} Éric Biernat et Michel Lutz. {\it Data science : fondamentaux et études de cas.} Eyrolles.
\end{thebibliography}

