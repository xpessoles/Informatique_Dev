%%%% Paramétrage du cours %%%%
\def\xxactivite{Cours}
\def\xxauteur{\textsl{Xavier Pessoles}}

\fichefalse
\proftrue
\tdfalse
\courstrue

\def\xxnumchapitre{Chapitre 1 \vspace{.2cm}}
\def\xxchapitre{\hspace{.12cm} Introduction}

\def\xxcompetences{%
\textsl{%
\textbf{Savoirs et compétences :}\\
\begin{itemize}[label=\ding{112},font=\color{ocre}] 
\item A voir
%\item \textit{Mod2.C17} : torseur dynamique
%\item \textit{Mod2.C17.SF1} : déterminer le torseur dynamique d’un solide, ou d’un ensemble de solides, par rapport à un autre solide
%\item \textit{Mod2.C15} : matrice d'inertie
%\item \textit{Res1.C2} : principe fondamental de la dynamique
%\item \textit{Res1.C1.SF1} : proposer une démarche permettant la détermination de la loi de mouvement
%\item \textit{Res1.C2.SF1} : proposer une méthode permettant la détermination d’une inconnue de liaison
\end{itemize}
}}



\def\xxfigures{
}%figues de la page de garde

\iflivret
\input{../../style/new_pagegarde}
\else
\input{../../style/new_pagegarde}
\fi
\setlength{\columnseprule}{.1pt}

\vspace{2cm}
\pagestyle{fancy}
\thispagestyle{plain}


\section{Définitions}

\begin{itemize}
\item Data scientist
\item machine learning
\item deep learning
\item réseaux de neurons
\item régression
\item data mining
\item bigdata
\item données continues, données discrètes, données nominales, données ordinales, données (semi-)structurées et non structurées

\end{itemize}

\begin{itemize}
\item algorithmes supervisés, non supervisés
\item algorithmes de régression et de classification
\end{itemize}

\begin{defi}[Machine Learning -- Apprentissage automatique -- Wikipedia] 
Champ d'étude de l'intelligence artificielle qui se fonde sur des approches mathématiques et statistiques pour donner aux ordinateurs la capacité d' « apprendre » à partir de données, c'est-à-dire d'améliorer leurs performances à résoudre des tâches sans être explicitement programmés pour chacune. 

La première phase de l'apprentissage consiste à estimer un modèle à partir de données, appelées observations, qui sont disponibles et en nombre fini, lors de la phase de conception du système.

 La seconde phase correspond à la mise en production : le modèle étant déterminé, de nouvelles données peuvent alors être soumises afin d'obtenir le résultat correspondant à la tâche souhaitée.
\end{defi}


\begin{defi}[Apprentissage supervisé -- Apprentissage non supervisé -- Apprentissage par renforcement -- Wikipedia] 
Si les données sont étiquetées (c'est-à-dire que la réponse à la tâche est connue pour ces données), il s'agit d'un apprentissage supervisé. On parle de :
\begin{itemize}
\item classification ou de classement si les étiquettes sont discrètes;
\item régression si elles sont continues.
\end{itemize}

Si le modèle est appris de manière incrémentale en fonction d'une récompense reçue par le programme pour chacune des actions entreprises, on parle d'apprentissage par renforcement. 

Dans le cas le plus général, sans étiquette, on cherche à déterminer la structure sous-jacente des données (qui peuvent être une densité de probabilité) et il s'agit alors d'apprentissage non supervisé.



\end{defi}

Exemples : 
\url{https://makina-corpus.com/blog/metier/2017/initiation-au-machine-learning-avec-python-pratique}

\begin{thebibliography}{2}
   \bibitem[1]{ref1} Éric Biernat et Michel Lutz. {\it Data science : fondamentaux et études de cas.} Eyrolles.
\end{thebibliography}

